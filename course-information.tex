\documentclass{tufte-handout}

\title{Graph Theory 1MA170: Course information}

\author[Vilhelm Agdur]{Vilhelm Agdur\thanks{\href{mailto:vilhelm.agdur@math.uu.se}{\nolinkurl{vilhelm.agdur@math.uu.se}}}}

\date{7 September 2023}

%\geometry{showframe} % display margins for debugging page layout

\usepackage{graphicx} % allow embedded images
  \setkeys{Gin}{width=\linewidth,totalheight=\textheight,keepaspectratio}
  \graphicspath{{graphics/}} % set of paths to search for images
\usepackage{amsmath}  % extended mathematics
\usepackage{booktabs} % book-quality tables
\usepackage{units}    % non-stacked fractions and better unit spacing
\usepackage{multicol} % multiple column layout facilities
\usepackage{lipsum}   % filler text
\usepackage{fancyvrb} % extended verbatim environments
  \fvset{fontsize=\normalsize}% default font size for fancy-verbatim environments

\usepackage{color,soul} % Highlights for text

% Standardize command font styles and environments
\newcommand{\doccmd}[1]{\texttt{\textbackslash#1}}% command name -- adds backslash automatically
\newcommand{\docopt}[1]{\ensuremath{\langle}\textrm{\textit{#1}}\ensuremath{\rangle}}% optional command argument
\newcommand{\docarg}[1]{\textrm{\textit{#1}}}% (required) command argument
\newcommand{\docenv}[1]{\textsf{#1}}% environment name
\newcommand{\docpkg}[1]{\texttt{#1}}% package name
\newcommand{\doccls}[1]{\texttt{#1}}% document class name
\newcommand{\docclsopt}[1]{\texttt{#1}}% document class option name
\newenvironment{docspec}{\begin{quote}\noindent}{\end{quote}}% command specification environment

\include{mathcommands.extratex}

\begin{document}

\maketitle% this prints the handout title, author, and date

\begin{abstract}
\noindent
This file is intended to contain all the practical information you may need about the course -- when is the exam, what is the assignment, when are the lectures, and so on.\sidenote{If you notice some information is missing, please do tell me and I will add it.}
\end{abstract}

The course literature for this course is Reinhard Diestel's \emph{Graph Theory}~\cite{Diestel_course_book}\sidenote{Which is available in a free pdf format from the university library:\\ \url{https://tinyurl.com/UUGraphTheoryBook}} and ``notes from the lecturer''. The course will of course be closer to the notes than to the book, but the book should still be a useful reference or alternative perspective -- though I do not guarantee that everything the lecture notes cover will be in the book. Since the course has a new lecturer this year\sidenote{The previous one defended his PhD and is no longer at Uppsala.}, ``notes'' could really refer both to the old notes and the new ones.

The previous year's lecture notes are already available on the course page -- the lecture notes from which I actually lecture will be made available as I create them, but they should generally be fairly close to what was done in the previous year.\sidenote[][]{In terms of subjects covered, they should be essentially equivalent. However, some examples might turn into exercises, and vice-versa, and notation will be slightly different between them.}

\section{Lecture plan}

There will be a total of twenty scheduled sessions, of which probably sixteen will be lectures, and the rest will be exercise sessions. The exercise sessions are important -- we will use them to introduce new concepts, and I \emph{will} assume in the lectures that you have been at the exercise sessions as well.\sidenote[][]{Of course, we all sometimes have to miss a lecture or exercise session. Attendance is, outside of the group project, voluntary. However, just as you would read the lecture notes to catch up on a missed lecture, you should attempt the exercises to catch up on a missed exercise session!} As stated in the hand-in assignment section, attendance in two of the exercise sessions is mandatory for that assignment.

The exact planning of the content of the lectures is still subject to change, in that some content may be reordered and some may be moved into exercises. The set of things covered is not going to differ much from the tentative plan given below, however.

\begin{table}[h]
\begin{tabularx}{\textwidth}{llX}
L\# & Time \& Date      & Content \\ 
\midrule
1  & 10:15, Nov 1 & E1: Introduction -- what is graph theory?\\
2  & 10:15, Nov 2 & L2: Eulerianity, simple graphs and subgraphs\\
3  & 08:15, Nov 3 & L3: Common graph families, trees, and Cayley's theorem\\
4  & 08:15, Nov 6 & L4: Spectral graph theory and the matrix-tree theorem\\
5  & 10:15, Nov 7 & E5: MSTs, flows, Hamilton cycles and independent sets\\
6  & 15:15, Nov 13 & L6: Weights, distances, and minimum spanning trees\\
7  & 10:15, Nov 15 & L7: The max-flow min-cut and marriage theorems\\
8  & 13:15, Nov 16 & L8: Vertex covers, Hamilton cycles, independent sets\\
9  & 10:15, Nov 20 & L9: ???\\
10 & 15:15, Nov 22 & L10: Connectivity\\
11 & 10:15, Nov 23 & L11: Planarity\\
12 & 10:15, Nov 27 & L12: Vertex colourings\\
13 & 15:15, Nov 29 & L13: More on colourings\\
14 & 15:15, Dec 4 & L14: Edge-colourings and Ramsey theory\\
15 & 10:15, Dec 6 & L15: ???\\
16 & 08:15, Dec 7 & L16: Szemerédi's regularity lemma\\
17 & 10:15, Dec 11 & L17: The Rado graph\\
18 & 15:15, Dec 12 & L18: The Erd\H{o}s-Rényi random graph\\
19 & 10:15, Dec 18 & L19: More on random graphs\\
20 & 10:15, Dec 20 & L20: ???
\end{tabularx}
\end{table}

\section{The exam}

The ordinary exam for the course is on the fifth of January 2024 -- remember to register at least twelve days in advance, i.e. by Christmas Eve\sidenote[][]{Some people get presents for Christmas, you get to register for an exam!}, in order to get to write it. Studying for an exam and then not getting to write it is pretty dispiriting.\sidenote{This may or may not have happened to me once or twice during my undergrad and master's...} The exam corresponds to 2hp out of the total of 5hp that the course consists of.

There will be reexams for the course on dates I will specify in this document as soon as I know them.

\section{The hand-in assignment}

The course has a \emph{mandatory} hand-in assignment, which corresponds to three out of the total of five hp of the course.

The assignment will be done in groups of four or five, and each group can pick between two tasks, either
\begin{enumerate}
  \item picking one of the lectures in the course, improving the lecture notes on it and writing up solutions to its attached exercises\sidenote[][]{The way to do this is to edit the \LaTeX\ source of the lecture notes -- come talk to me after a lecture for practical instructions on how to do this. It is much easier to just do it in reality than try to write a step-by-step guide here.}, or
  \item reading a research paper in graph theory or network science and writing a short (no more than five pages) summary of what problem the paper addresses and the methods they use.
\end{enumerate}

You will be required to show up at one of the exercise sessions before you hand in your assignment to work on it, and to show up at an exercise session after you hand it in in order to talk about your work with me.\sidenote[][-0.6cm]{Partially, this is of course in order to prevent you from using ChatGPT or similar to cheat on the assignment. That is, however, not the main reason -- the main reason is to be able to help you with your work on it, and to be able to give better feedback on it afterwards. In-person feedback is a hundred times better than written feedback, and the goal of an assignment should be primarily that you learn from doing it -- the need for assessment is secondary.} The deadline for the assignment is two weeks before the final exercise session, if you pick the paper, and two weeks after the lecture if you choose to work on one of the lectures.\sidenote[][2.7cm]{If you pick a lecture very near the start or the end of the course we may have to adjust the deadlines slightly to fit within the course and with the exercise session requirement -- come talk to me about it if you need an adjustment of the deadline.}

If you have a reasonable reason, we can extend your group's deadline \emph{if you request this extension before the deadline}. Extension requests after you have already exceeded the deadline are much more difficult to grant.

\bibliography{references}
\bibliographystyle{plainnat}



\end{document}
